\chapter*{Abstract}
\addcontentsline{toc}{chapter}{Abstract}

%This thesis is concerned with the testing of surface support elements under dynamic loading conditions. 
%Firstly there's some literature and definitions and some previous tests by other people are described.
%Secondly the test rig and test method and its limits are described.
%Thirdly results of the tests are presented and discussed.
%Lastly the best possible surface support structure is discussed in detail and recommended further steps are given.

In order to optimize the energy absorption capabilities of the surface support system, Kiirunavaara mine, owned by LKAB, has constructed a drop test rig. It has a drop height of up to 4 m and a maximum drop weight of 1000 kg. Load cells, laser distance measurements, high speed camera footage and multiple accelerometers ensured comprehensive data collection.

It is capable of testing round discrete concrete samples with a diameter of 0,8 m (after \textcite{c1550}) and square concrete samples with an edge length of 1,5 m. The square samples are cast on top of large concrete slabs that serve to simulate the adhesion between shotcrete and rock. It is possible to test different types of mesh with varying boundary stiffness and rate of pretension as well as different types of lacing and straps. Concrete, mesh and lacing can be freely combined, therefore it is possible to test various surface support concepts. 

Several tests were run with this test rig. The tests were focused on optimising the rig and all changes that were made are discussed in detail. From the results of the first tests is was possible to devise equations that allow to estimate the energy absorption capabilities of shotcrete at different thicknesses.

Later campaigns will test the support system currently in use at Kiirunavaara Mine as well as some modifications of it.
%The results of these test are presented and discussed in detail.

Lastly future modification that could be done to make the rig more comprehensive are discussed, for example being able to conduct dynamic tests on bolts, quasi-static tests on surface support or dynamic tests on complete support systems. %It also provides some ideas for future test matrices.
Some ideas for future test matrices are given.

%%%%%%%% TO DO Talk about results

%The goal of the first test series run with the rig was to find potential enhancements and increase the credibility of the results of further tests. Round samples were used for this test.

%When the rig had been modified sufficiently, tests on large square samples were conducted. The tested materials included: fibre reinforced concrete, welded mesh, chain-link mesh, lacing, straps ("Fjällband"). The boundary conditions of the tested meshes were considered at length.



%A small series of tests was run on round samples with a diameter of 800 mm. This report discusses the findings of those tests and the potential of the rig.

%The results of the test show quite clearly that the test procedure would benefit from functioning way of measuring displacement and that the energy absorption capabilites rise with the thickness of the concrete panel, as expected. 

%The currently used method of measuring displacement by a laser sensor suffered some complications. The way displacement is measured will have to be reconsidered. The analysis of the test data was greatly affected by the lack of displacement data, therefor not many conclusions could be drawn. It mainly became apparent that the energy absorption capabilities of the test panels rose with the thickness of the panels.

%Using the findings of this report the testrig will be optimized so that the tests on the large samples will have good results.

%\chapter*{Zusammenfassung}
% \addcontentsline{toc}{chapter}{Zusammenfassung}