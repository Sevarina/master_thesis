\chapter{Discussion}
\label{ch:con}

%Why was it necessary to write this thesis?
%What were the results?
%What are the next steps?
%In brevity.

With increasing depth, stresses inside the rock mass surrounding the stope increase. When these stresses surpass the strength of the rock, it is bound to fail. In case of strong, brittle rock, which shows little ductile behaviour, tensions cannot be reduced through continuous deformations. These tensions build up and can be released explosively. %, rocks might get ejected from the walls and ceiling into the cavity.
Such an event is called a rockburst. They are often connected to seismic events of natural or artificial origin. An example for a seismic event of artificial origin is blasting. As blasting is a regular occurrence in most underground mines and developments in technology allow for economical exploitation of ore bodies located deeper than ever before, rockburst are an extremely relevant topic.

Rockburst occurrence and severity can be reduced through various methods, for example: changing or modifying production sequence, mining geometry, mining method, excavation shape, location of drifts and ore passes, and stope size. These measures can reduce the stress on the rockmass in advance. Destress drill and blasting can be used to actively reduce the likelihood of a rockburst in a limited area. \autocite[219]{Kaiser12}

The last line of defence against rockbursts is rock support. There are two kinds: reinforcement support and structural support. 
Reinforcement support is usually realized through rock bolts. There are several kinds with some being especially designed to take the dynamic loads caused by rockbursts. Reinforcement support aims to increase the strength of the rock mass to the point at which it can support itself. 
Surface support is more varied than reinforcement support: shotcrete (not reinforced, fibre reinforced, mesh reinforced), steel mesh, lacing, straps and combinations of these all function as structural support. Surface support together with reinforcement support forms a safety net that reduces the risk of rock falls by capturing loose rock or securing wedges which would be prone to fall by anchoring them to unbroken rock. As the weakest link of a support system will fail first, a support system must be considered holistically.

Without the danger of rockbursts, such rock support systems are designed to withstand static load. The change in pressure of the rock is slow, resulting in quasi-static loading conditions. In burstprone ground  sudden high energy bursts must be absorbed by the support system as well. %This is usually achieved by allowing deformations, channeling energy into friction or other ways of turning the destructive energy into less harmful effects. 
A system cannot be designed for the sole purpose of absorbing energy. It must also withstand the static load imposed by the surrounding rock mass. \autocite[233, 234]{Heal10} 

Developing fitting rock support systems is difficult as they must be adjusted to local conditions and few practical test methods are available, even fewer of them standardized. Taking data from past rockbursts and gathering data about energy levels of past seismic events, their locations and how the rock support reacted to them is one important source of information. The limitation of this information is, that only the current support system can be evaluated. If a new support system would be tested this way it would have to be deployed in the mine at least partially.
%%% REPLACE
%But unless new rock support systems are deployed in the mine no data on them can be had that way.
%%%SUGGESTIONS
%Only data about the current rock support system is gathered when using this system. A new system would have to be deployed at least in parts of the mine to be able to test it this way.
%%%%
But just using a new support system without knowing anything about it and to get data through observation of its reaction to real events could lead to significant damage to excavation, material and most importantly: humans.
That is why a rock support system should be tested before it is used. For this purpose it's possible to host \textit{in - situ} tests, where the support system is deployed and through controlled blasting its energy absorption potential is studied. Such tests are extremely costly and time intensive and cannot easily be replicated. Another approach therefore is the construction of rigs which allow for easily replicable test of large amounts of different support systems quickly and cheaply. Most of them work by dropping a weight on the sample that is to be tested, and then studying the reaction.


Rockbursts are a growing problem at the Kiirunavaara Mine. The support concept implemented in burst prone areas consists mostly of fibre reinforced shotcrete, covered with mesh and dynamic bolts, with a type of strap called Fjällband being deployed in some special cases. 
To be ready for the future and the increasing rock stresses at deeper levels a drop test rig was constructed to quickly and cheaply run large amounts of tests on many different combinations and have comparable results.

With the current test rig, it was possible to run a few tests in within an hour when using minimal instrumentation. Therefore, the bottleneck when testing large sets of specimen is the sample production time. Especially concrete specimens require a long lead time to cure. Another strength of the test rig built in Kiirunavaara Mine is the ability to quickly test large batches of small samples that are easy and cheap to manufacture to gain some basic insights into the material. The most promising candidates of those tests can then be followed up with more comprehensive and realistic tests on significantly larger samples on the same rig.
A high test rate is positive, as \textcite[14]{Crompton18} puts it: "The ability to conduct a large number of tests provides the opportunity to rapidly increase the knowledge base on the dynamic performance of support elements." and was of high priority during the design process of the conducted experiments.
The tests on large samples on the other hand allow the user to make more realistic predictions of the \textit{in - situ} behaviour of the rock support.

This first tests were run mostly to gain practise in using the test rig, to stream line the testing process and to gain some understand of the influence of shotcrete thickness to energy absorption capabilities. FRS thickness and energy absorption appeared to be linearly dependant when controlling for various variables. Equations were formulated to quickly estimate the FRS thickness necessary to survive a given impact energy.

One of the first challenges that poses itself when conducting dynamic tests is defining a failure criteria. With the current test rig the deformations are very restricted. If the rig is changed the failure criteria will have to be reevaluated.

The sensors chosen for this test rig were chosen based on their resolution and their usability. The rig appears to work as planned, the measured values correlate well with each other. There a very few measurements taken directly from the sample which means that there is little information about the sample itself behaves. %The thickness of the sample does not appear to correlate in a linear fashion with any other parameter. It seems like thicker concrete can take high impact energy. 
For some tests additional accelerometers were applied and they showed that the sample experiences significantly larger acceleration than the drop weight and appears to move horizontally. The additional accelerometers had a negative effect on the test results. Therefore they should not be used for any more tests.
The drop weight appears to bounce after the impact while the sample appears to settle quickly into its new position. The problems with the laser sensor can be ameliorated by using reflective tape on the sample. The displacement and the formation of cracks appears not to correlate with the absorbed energy. Samples can only be divided into the categories "broken" and "cracked" with any certainty. 
The test rig at Kiirunavaara yields very conservative results for FRS. The reason for this is unclear.

It is difficult to estimate the post peak capabilities of a support system. Maybe quasi-static tests could help with this or at least compliment the information gained from dynamic tests.

%The sensor array should large enough to gather the important data but not bulky or bloated. The decision which data is important is difficult one. There are two basic design paradigms.

%Either a few very well instrumented tests that as realistic as possible 

The tests run for this thesis had the aim of setting a baseline and determining what are the important factors at play. Follow up test can use this knowledge to decide which data to exclude from the measurements to stream line preparatory work and post processing. 

%Each sensor already in the array should be questioned as well.
%Is it possible to turn the data it yields into information? Can we use this information to create knowledge?


%A larger amount of simpler tests could yield more interesting results than a few heavily instrumented ones. 
%It is simply a question of what the aim of these tests is. 
%Is it important to run a few very well instrumented test and gather a lot of information about how each individual sample reacts and behaves and analyse them in depth. Or is it more important to run a lot of lean tests with minimal instrumentation and superficial analysis and try to build a statistical model of the most important factors. 

%Is it important to know exactly how the support system behaves or is an approximation good enough.

%The test results were difficult to interpret as vital measurements were missing. Before further tests are run it is of critical importance that a reliable way of measuring displacement is found. Otherwise the time and capital invested in specimen production is wasted.

More drop tests will be necessary to find a support system that is optimised for the Kiirunavaara Mine use case and be adapted for the other large underground mine that LKAB owns: Malmberget.

The next step after sufficient amounts of drop tests will be \textit{in - situ} tests that study both the usability and the behaviour and properties under real conditions, lastly the new system will be deployed and back calculations will reveal how much it has improved mine safety. 
